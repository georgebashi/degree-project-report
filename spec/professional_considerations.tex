\section{Professional Considerations (612/500)}
% Professional considerations - this should outline any professional ethical issues which impinge upon your project. The ethical standards governing the computing profession in Britain are defined by the Code of Conduct and Code of Practice published by the British Computer Society (www.bcs.org). You should consult these for guidance on ethical issues, no matter how abstract or scientific your project is. For project topics which are affected only slightly by ethical issues, it is sufficient to discuss them in the project introduction.

% outline any professional ethical issues which impinge upon your project (www.bcs.org)
% ethical issues - say how they apply, in relation to it being abstract or scientific
\subsection{Copyright}
As the system deals with the use of music recordings, 
It is important to consider the legal implications of using music recordings in a context other than listening, as the system will be dealing with the analysis and feature extraction a large number of musical works, where the stored feature vectors could be seen as `derivative works'. The British Computer Society's (BCS) Code of Conduct states that ``you shall ensure that within your professional field/s you have knowledge and understanding of relevant legislation, regulations and standards, and that you comply with such requirements''. The ``relavent legislation'' in this context is the Copyright, Designs and Patents Act (1988), which has a clause exempting the use of music for research purposes: ``Fair dealing with a literary, dramatic, musical or artistic work for the purposes of research or private study does not infringe any copyright in the work or, in the case of a published edition, in the typographical arrangement.''

My use of music recordings in analysis and user evaluation are considered ``fair dealing'' due to this clause and as I
will only be providing a system for analysis and indexing of the user's content,
the onus is on the user themselves to obtain their digital music in a legal fashion.

\citet{Sturm2006} wrote a pertinent paper on the topic of music analysis and storage, ``Concatenative Sound Synthesis and Intellectual Property: An Analysis of the Legal Issues Surrounding the Synthesis of Novel Sounds from
Copyright-Protected Work''. This paper, concludes that
``in creating corpus databases for use in CSS [Concatenative Sound
Synthesis, involving storing fragments of a song in a database and
using this to create new sounds], the underlying data representing a
sound may not change, but its function is transformed from one of
entertainment to one of supplying statistical data for driving sound
synthesis. As long as the corpus database is not distributed or sold,
its incorporation of material protected by copyright is defensible''.
Likewise, the system will be generating features for similarity measure, transforming the function of the ``derivative work'', and it is these that are stored on disk: ``[if] the defendant's work, although containing
substantially similar material, performs a different function than that of the plaintiff's, the defence of fair use can be invoked'' \citep{Copyright1988}.
\subsection{Competence}
On a more general level, it is important to undertake the project with good professional conduct. I will aim to adhere to the BCSs Code of Good Practice and Code of Conduct, especially the cause which states: ``you shall not claim any level of competence that you do not possess. You shall only offer to do work or provide a service that is within your professional competence.''
% where my study during my degree lies
This project involves the study of Computational Music, one of the core courses
from degree I have studied, and draws on knowledge gained from other courses such as
Data Mining, Artificial Intelligence and Multimedia Design and Applications.
As I have chosen a topic in the area of Music Informatics, I have the level of
competence required to undertake the project, and am able to research anything
which falls outside my area of knowledge.
\subsection{Falsifying of evidence}
% quote bcs
% my results are fully repeatable
It is important in any paper to not falsify any results or output of the system or topic discussed. From the BCS Code of Good Practice: ``You shall not misrepresent or withhold information on the performance of products, systems or services, or take advantage of the lack of relevant knowledge or inexperience of others.'' As contracted in the declaration (inside cover), I will show no examples of falsification of evidence, and present all findings in full. In addition, full source code and an explanation of its use is given, and so all tests are fully repeatable by a third party.
