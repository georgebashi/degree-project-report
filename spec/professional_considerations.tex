\section{Professional Considerations (500)}
% Professional considerations - this should outline any professional ethical issues which impinge upon your project. The ethical standards governing the computing profession in Britain are defined by the Code of Conduct and Code of Practice published by the British Computer Society (www.bcs.org). You should consult these for guidance on ethical issues, no matter how abstract or scientific your project is. For project topics which are affected only slightly by ethical issues, it is sufficient to discuss them in the project introduction.

% outline any professional ethical issues which impinge upon your project (www.bcs.org)
% ethical issues - say how they apply, in relation to it being abstract or scientific
\subsection{Copyright}
As my project involves the use of music recordings, it is important to consider the legal implications of using these in a context other than listening. As stated in the Copyright, Designs and Patents Act 1988:
\begin{quotation}\noindent
29. (1) Fair dealing with a literary, dramatic, musical or artistic
work for the purposes of research or private study does not infringe
any copyright in the work or, in the case of a published edition, in
the typographical arrangement.\\
(2) Fair dealing with the typographical arrangement of a published
edition for the purposes mentioned in subsection (1) does not infringe
any copyright in the arrangement.\\
(3) Copying by a person other than the researcher or student himself
is not fair dealing if ---
\begin{quotation}\noindent
(a) in the case of a librarian, or a person acting on behalf of a
librarian, he does anything which regulations under section 40 would
not permit to be done under section 38 or 39 (articles or parts of
published works: restriction on multiple copies of same material), or\\
(b) in any other case, the person doing the copying knows or has
reason to believe that it will result in copies of substantially the same
material being provided to more than one person at substantially the
same time and for substantially the same purpose.
\end{quotation}
\end{quotation}
Due to this clause, my use of music recordings for the research (such as user
evaluation using short clips of audio etc.\ ) are considered ``fair dealing'', and as I
will only be providing a system for analysis and indexing of the user's content,
the onus is on the user to obtain their digital music in a legal fashion. Also, there is no copying of the music (just analysis and playback of the recording)
and so subsection 3 is not invoked.

As the result of the analysis process, feature vectors will be obtained from the
audio (which could be argued are ``derivative works''), and stored in a database
alongside a reference to the specific track. Bob Sturm wrote a particularly
pertinant paper, ``Concatenative Sound Synthesis and Intellecual Property: An
Analysis of the Legal Issues Surrounding the Synthesis of Novel Sounds from
Copyright-Protected Work''[5]. In this paper, he concludes that:
In creating corpus databases for use in CSS [Concatenative Sound
Synthesis, involving storing fragments of a song in a database and
using this to create new sounds], the underlying data representing a
sound may not change, but its function is transformed from one of
entertainment to one of supplying statistical data for driving sound
synthesis. As long as the corpus database is not distributed or sold,
its incorporation of material protected by copyright is defensible.
Likewise, in my system I will be generating feature vectors for similarity matching, transforming the function of the ``derivative work'', and it is these that
are stored in the database — ``[if] the defendant's work, although containing
substantially similar material, performs a different function than that of the
plaintiff's, the defense of fair use can be invoked'' (REF!)

On a more general level, it is important to undertake the project with good
professional conduct. I will aim to adhere to the British Computer Society's
Code of Good Practice and Code of Conduct, especially the following sections:
15. You shall not claim any level of competence that you do not
possess. You shall only offer to do work or provide a service that is
within your professional competence.
\begin{quotation}\noindent
You can self-assess your professional competence for undertaking a particular job or role by asking, for example,
\begin{quotation}\noindent
i. am I familiar with the technology involved, or have I worked with
similar technology before?\\
ii. have I successfully completed similar assignments or roles in the
past?\\
iii. can I demonstrate adequate knowledge of the specific business
application and requirements successfully to undertake the work?
\end{quotation}
\end{quotation}
\subsection{Competence}
% where my study during my degree lies
My project involves the study of Computational Music, one of the course courses
from my degree, and draws on knowledge gained from other courses such as
Data Mining, Artificial Intelligence and Multimedia Design and Applications.
As I have chosen a topic in the area of Music Informatics, I have the level of
competence required to undertake the project, and am able to research anything
which falls outside my area of knowledge.

The BCS standards also state that I should adhere to all relevant laws:
``You shall ensure that within your professional field/s you have
knowledge and understanding of relevant legislation, regulations and
standards, and that you comply with such requirements''.
By not copying music, and for the reasons previously discussed, I will not be
infringing any copyrights.
\subsection{Results}
\subsubsection{Falsifying of evidence}
% quote bcs
% my results are fully repeatable
\begin{quotation}
You shall not misrepresent or withhold information on the performance of products, systems or services, or take advantage of the lack of relevant knowledge or inexperience of others.
\end{quotation}

