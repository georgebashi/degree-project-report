% Body of report - this should include a requirement analysis and specification of the problem you have tackled.
%
%%%%%%%%%%%%%%%%%%%%%%%%%%%%%%%%%%%%%%%%%%%%%%%%%%%%%%%%%%%%%%%%%%%%
%                      JUSTIFY ALL MY CHOICES                      %
%%%%%%%%%%%%%%%%%%%%%%%%%%%%%%%%%%%%%%%%%%%%%%%%%%%%%%%%%%%%%%%%%%%%
%
\chapter{Specifications and Requirement Analysis (2000)}
\section{Specifications}
% set out what my software has to do
\begin{itemize}
	\item Why specify system behaviour
	\item Why it helps
	\item Relate back to software review
	\item Overview
\end{itemize}
\begin{itemize}
	\item Analyse the user's media library in a feaseable time-frame (overnight)
	\item Avoid storing too much data (slow, uses lots of disk space etc)
	\item Provide a simpler method of playlist generation than selecting individual tracks
	\item Base playlist generation purely on analysis of audio data
\end{itemize}
\subsection{Extensions}
\begin{itemize}
	\item Provide GUI
	\item Run on portable devices
\end{itemize}
\section{Requirement Analysis}
By looking at the core functionality of the application itself,
I can set out more specific requirements about the software.
\subsection{Software \& Library Requirements}
\begin{itemize}
	\item Give background on libraries used and what they are used for
	\item Give alternatives to some libraries and explain why they wern't used (Marsyas and MAAATE)
	\begin{itemize}
		\item lack of knowledge of libraries - would have taken as long to learn it as to have written it myself
		\item too big to leverage into the proof-of-concept example needed
		\item wouldn't allow as much flexibility in operation of the program
		\item this way is easier to modify in future to add more extractors / build on in any way
		\item why they would have been good
		\begin{itemize}
			\item easier to chop and change between features (not important for my project)
			\item ?
		\end{itemize}
	\end{itemize}
	\item GStreamer (vs. libmad, libogg/vorbis, libflac; directsound)
	\item libsndfile
	\item FFTW3 (vs. hand coded - OMG)
	\item libpopt (vs. hand coded - just lazy)
\end{itemize}
\subsection{Usability Requirements}
\begin{itemize}
	\item Interface
	\begin{itemize}
		\item commandline as its simplest to write
		\item guis can easily be written to use the commandline tools in the background
	\end{itemize}
	\item Storage
	\begin{itemize}
		\item Seperate files easier to code \& debug
		\item Database slightly less reliable (moved files need to be recalculated) but much faster \& tidier
		\item Data stored on each track needs to be small so as not to take too much disk space
	\end{itemize}
\end{itemize}
% It should also include a description of how you have designed, built and evaluated your system.
% The exact form of this will vary from project to project but it will usually occupy several chapters
% and will often include sections on implementation and testing. Any software projects should included a 
% discussion of the principles which underlie the program which has been written: the significance of its data
% structures, the way that its procedures and modules interact etc.


