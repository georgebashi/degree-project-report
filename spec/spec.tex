% Body of report - this should include a requirement analysis and specification of the problem you have tackled.
%
%%%%%%%%%%%%%%%%%%%%%%%%%%%%%%%%%%%%%%%%%%%%%%%%%%%%%%%%%%%%%%%%%%%%
%                      JUSTIFY ALL MY CHOICES                      %
%%%%%%%%%%%%%%%%%%%%%%%%%%%%%%%%%%%%%%%%%%%%%%%%%%%%%%%%%%%%%%%%%%%%
%
\newcommand{\objective}[1]{
	\subsubsection{#1}
}
\chapter{Specifications and Requirement Analysis (1789/2000)}
\label{text:spec}
\section{Overview}
In order to investigate music similarity measure and its use in interpolative playlist generation, it is important to write a proof-of-concept piece of software which is capable of performing these functions, to show that the concepts discussed and the method proposed and designed are solid. A detailed specification of system behaviour has been designed in order to focus the development of this system; as stated in \citet{Bourque2004}, ``The Software Requirements Knowledge Area is concerned with the elicitation, analysis, specification, and validation of software requirements. It is widely acknowledged within the software industry that software engineering projects are critically vulnerable when these activities are performed poorly''. Specifying what is required of the software and the requirements of the software itself helps to design a system based around those goals and hence have a more succinct design and implementation. Additionally, later comparing the final system back to these specifications allows for critiquing of the system in terms of how well it met its goals and  assessing the viability of the system (or a future system based on the technique) for development into an end-user application.
\section{System Objectives}
\subsection{Research Requirements}
The specification of the system begins with the requirements of the software developed for research purposes. 
\objective{Provide a method to verify and view results at each step of the processing}
Verification and the ability to view input and output data of the system is important during the development process in order to ensure the correct operation of the code and to track down any software errors which may arise. Tracking data flow through each step of the system can help to show any areas which may be operating incorrectly or sub-optimally.

To achieve this, the system will be designed so that it is easy to debug, as well as providing a tool to view any binary files, such as the output of the feature extractor, which may be stored in order to verify correct operation. Also, where necessary, program options will be created which will reveal more detailed output about the operation of the system for testing and debugging purposes.
\objective{Provide a simple method to analyse music and create playlists in batch for testing}
% reading compressed files
\label{text:spec:objective:cli}
As the user's media library containing a large amount of audio files will need to be processed and a number of playlists generated for testing and debugging, it is important to have an interface to the program which allows for this. A command-line interface has been chosen, with the mode of operation and the files to process specified on the command-line and all output printed back to console in a standard playlist format, allowing playlists to be directly loaded into a media player. This simplifies the code needed to process options, and allows for batch scripts to be created which can automate testing. Also, operating without a graphical user interface (GUI) eases debugging of the system, which is important for the above requirement.
\objective{Allow components to be added and modified without (severely) affecting the operation of others}
% Encapsulation / Abstraction
Separating the concerns of similarity measure (audio processing and analysis) and playlist generation (statistical analysis) into separate programs means that the code itself will be better structured and hence easier to comprehend, which in turn means that it is easier to understand, develop, and test. As the algorithm for the similarity metric and playlist generation will be modified frequently during the development of the system, it is important that the two do not depend on each other more than necessary. Following the Object-Oriented design philosophy and its ideas of abstraction and encapsulation the software will be written in \software{C++}, and the program logic separated into classes, each dealing only with their own data and processing.
\subsection{User Requirements}
In addition to the development and research requirements, it is important to consider the needs of the hypothetical end-user, who would use a system based upon the technique later discussed. By setting out the requirements for a viable playlist generating application, the system and technique can be designed around this. The following are a set of base-line requirements for a feasible playlist generation system.
\objective{Analyse the user's media library in a reasonable time-frame}
Digital signal processing and audio analysis such as required by any audio-based similarity measure requires a great deal of data processing --- audio must be read from disk byte-by-byte and calculations performed based on this --- if a typical song (uncompressed) is 20 megabytes (MB) this translates to 5,242,880 floating-point operations (which are very computationally expensive) just to calculate the mean value! As the processing done on each song is immense, it is important to have this processing run as quick as possible and avoid the need to re-process any one song.

To achieve the kind of speed needed, \software{C++} has been chosen as the development language as it is very low-overhead and suited to intensive processing. As the amount of calculation required is unavoidable, the need to re-calculate data for a file will be avoided by storing the analysed data along with the audio files themselves; this means that if files are moved, the analysis data is moved with them.
\objective{Avoid storing too much data}
\label{text:spec:requirement:data}
If a large amount of data for each song is stored it affects the scalability of the application, as for large media libraries the analysis data itself will take up a large amount of disk space. Likewise, the more data stored, the more calculations needed to compute similarity, increasing the time taken to generate playlists exponentially.
\objective{Provide a simpler method of playlist generation than selecting individual tracks}
In current media player software, playlists can be created by the user by adding tracks one-by-one to a queue. This is a slow and laborious process, requiring the user to make decisions about each track they would like to hear, retrieve it from the library and add it to the queue. The system designed will help to alleviate this by allowing the user to create playlists on the macro level; the user picks a set of key tracks, and the system `fills the gaps' between them to create a smooth transition of music. This allows the user to still specify and be in control of the mood and genre of the playlist, yet not having to choose each individual song, thereby creating a much easier-to-use and less time-consuming method of generating a desired playlist.
\objective{Base playlist generation solely on analysis of audio data}
Further to the review of current software implementations, the decision has been made to use a sub-symbolic similarity metric so there is no reliance on pre-existing metadata for each song in the user's media library.
\objective{As an extension, provide a GUI for ease of use}
If the system designed performs well enough to be developed into an end-user application, the command-line interface would need to be re-worked into a GUI for ease of use. Although the GUI lends itself to testing and development, the average home user lacks the technical proficiency to use a command-line application. Providing a GUI would allow users to select tracks and generate playlists with ease, further meeting the above requirement of easing the burden of playlist generation. As this would need a major re-work of the system, it has been drafted as an extension for possible future work.
% The system which I have designed and created is a small demonstration system, tailored for ease and speed of development as development time is limited, verifiability of results to ensure correct operation of the system, and ease of testing outputs to assess the overall performance of the system.
\section{Requirement Analysis}
In addition to analysing what is required of the system, it is important to look at what will be required by the system itself. After looking at the requirements and functionality set out above, a list of software requirements of the system has been outlined.
\subsection{Software Libraries}
Throughout the development of the system, extensive use of software libraries has been made where possible to speed development time and reduce errors. As there are many software libraries available to aid in any individual task, several options have been evaluated and reasons given for each choice.
\subsubsection{Audio Decoding}
Digital audio, when converted to analog for playback or processed in some way is read as a continuous stream of floating-point numbers between 1 and -1, representing the current level of the waveform. When stored on disk, however, there are hundreds of file formats, each with a different set of features and an entirely different algorithm for obtaining the floating-point numbers needed for processing. In a typical user's media library, there can be audio of very different types: MP3 for portable devices, AAC on Apple machines, WMA on Windows, FLAC for lossless compression, WAV for uncompressed audio; it is important that any signal analysis system is able to process at least these types in order to be usable. For the purposes of this proof-of-concept code, the system was initially developed using \software{libsndfile}\footnote{\url{http://www.mega-nerd.com/libsndfile/}}, which can read uncompressed audio files (such as WAV and AIFF). While this was suitable for development work, it quickly became evident that if a large number of audio files were to be processed, the system would need to be able to decode compressed audio (mainly MP3 and FLAC in my test corpus). In order to overcome this, the \software{GStreamer} multimedia framework has been used to decompress, down-sample and mix audio to mono (\codepageref{SoundFile.cc}{21}) into a temporary WAV file, ready to be read in by libsndfile (\coderef{SoundFile.cc}{29}). While this provides a simple solution to the problem, it is not optimal as it induces the large overhead of spawning another process and writing to a temporary file; again a trade-off with development time and optimality, reading from the major formats could be directly implemented using their individual libraries, gaining processing speed and reducing overhead.
\subsubsection{Audio Analysis / DSP}
For the audio analysis section of the system, it was possible to use a number of software libraries to implement the signal processing and analysis. \software{Marsyas}\footnote{\url{http://marsyas.sf.net/}}, \software{MAAATE}\footnote{\url{http://www.cmis.csiro.au/maaate/}} and \software{libaudiofeat}\footnote{\url{http://www-etud.iro.umontreal.ca/~bergstrj/doc/audiofeat/html/index.html}}  are three libraries which allow the developer to pass in audio data and receive analysis data. Although this would have avoided the need to directly implement feature extractors (discussed later), it would mean sacrificing both speed in unneeded calculations (\software{libaudiofeat}) and flexibility in the feature extractors (\software{Marsyas} and \software{MAAATE}) which could be used. Also, as these were large and complex libraries, it would have taken as long to learn how to use the library as it would write to the code from scratch; due to this, the feature extractors were manually implemented and tested.
The feature extraction process required the computation of a Fourier Transform; as this is a complex mathematical procedure, the  \software{FFTW3} library has been used to perform the computation as it is a highly optimised library specifically for doing Fourier Transforms.
\subsubsection{User Interface}
As discussed previously (System Objectives, page \pageref{text:spec:objective:cli}), a command-line interface has been implemented. In order to parse the options specified on the command-line, the \software{libpopt} library has been used to provide a unified interface to the user and allow for simple batch scripting. Although this could have been implemented manually, much of the code of \software{libpopt} would need to be duplicated and would provide no real benefit.
\begin{comment}
\subsection{Hardware \& Storage}
\begin{itemize}
	\item Hardware
	\begin{itemize}
		\item Fast processor
		\item Lots of HDD space for music itself, not much on top of that
	\end{itemize}
	\item Storage
	\begin{itemize}
		\item Seperate files easier to code \& debug
		\item Database slightly less reliable (moved files need to be recalculated) but much faster \& tidier
		\item Data stored on each track needs to be small so as not to take too much disk space
	\end{itemize}
\end{itemize}	
\end{comment}


