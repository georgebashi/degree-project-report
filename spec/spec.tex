\chapter{Specifications and Requirement Analysis (2000)}
\section{Specifications}
% set out what my software has to do
Analyse the user's media library in a feaseable time-frame (overnight)\\
Provide a simpler method of playlist generation than selecting individual tracks\\
Base playlist generation purely on analysis of audio data
\subsection{Extensions}
Provide GUI\\
Run on portable devices
\section{Requirement Analysis}
By looking at the core functionality of the application itself, I can set out more specific requirements about the software.
\subsection{Storage Requirements}
BDB vs data files\\
Size constraints
\subsection{Software / Library Requirements}
Audio decoder (extension to include compressed audio)\\
FFTW3
\subsection{Usability Requirements}
Commandline interface - simplest to write and guis can easily be written to use the commandline tools in the background\\
Tidier to store things in DB
% Body of report - this should include a requirement analysis and specification of the problem you have tackled. It should also include a description of how you have designed, built and evaluated your system. The exact form of this will vary from project to project but it will usually occupy several chapters and will often include sections on implementation and testing. Any software projects should included a discussion of the principles which underlie the program which has been written: the significance of its data structures, the way that its procedures and modules interact etc. A line by line description of your code is not necessarily the best way of achieving this and is not encouraged.
\section{Professional Considerations (500)}
% Professional considerations - this should outline any professional ethical issues which impinge upon your project. The ethical standards governing the computing profession in Britain are defined by the Code of Conduct and Code of Practice published by the British Computer Society (www.bcs.org). You should consult these for guidance on ethical issues, no matter how abstract or scientific your project is. For project topics which are affected only slightly by ethical issues, it is sufficient to discuss them in the project introduction.

% outline any professional ethical issues which impinge upon your project (www.bcs.org)
% ethical issues - say how they apply, in relation to it being abstract or scientific
\subsection{Copyright}
As my project involves the use of music recordings, it is important to consider the legal implications of using these in a context other than listening, as is stated in the British Computer Society's Code of Conduct: ``You shall ensure that within your professional field/s you have knowledge and understanding of relevant legislation, regulations and standards, and that you comply with such requirements''. The Copyright, Designs and Patents Act 1988 states that:
\begin{quotation}
	\renewcommand{\labelenumi}{\arabic{enumi}}
	\renewcommand{\labelenumii}{(\arabic{enumii})}
	\renewcommand{\labelenumiii}{(\alph{enumiii})}
	\begin{enumerate}
	\setcounter{enumi}{28}
		\item \begin{enumerate}
			\item Fair dealing with a literary, dramatic, musical or artistic work for the purposes of research or private study does not infringe any copyright in the work or, in the case of a published edition, in the typographical arrangement.
			\item Fair dealing with the typographical arrangement of a published edition for the purposes mentioned in subsection 1 does not infringe any copyright in the arrangement.
			\item Copying by a person other than the researcher or student himself is not fair dealing if:
			\begin{enumerate}
				\item in the case of a librarian, or a person acting on behalf of a librarian, he does anything which regulations under section 40 would not permit to be done under section 38 or 39 (articles or parts of published works: restriction on multiple copies of same material), or
				\item in any other case, the person doing the copying knows or has reason to believe that it will result in copies of substantially the same material being provided to more than one person at substantially the same time and for substantially the same purpose.
			\end{enumerate}
		\end{enumerate}
	\end{enumerate}
\end{quotation}
Due to this clause, my use of music recordings for the research (such as user
evaluation using short clips of audio etc.\ ) are considered ``fair dealing'', and as I
will only be providing a system for analysis and indexing of the user's content,
the onus is on the user to obtain their digital music in a legal fashion. Also, there is no copying of the music (just analysis and playback of the recording)
and so subsection 3 is not invoked.

As the result of the analysis process, feature vectors will be obtained from the
audio (which could be argued are ``derivative works''), and stored in a database
alongside a reference to the specific track. Bob Sturm wrote a particularly
pertinant paper, ``Concatenative Sound Synthesis and Intellecual Property: An
Analysis of the Legal Issues Surrounding the Synthesis of Novel Sounds from
Copyright-Protected Work'' \citep{Sturm2006}. In this paper, he concludes that
``in creating corpus databases for use in CSS [Concatenative Sound
Synthesis, involving storing fragments of a song in a database and
using this to create new sounds], the underlying data representing a
sound may not change, but its function is transformed from one of
entertainment to one of supplying statistical data for driving sound
synthesis. As long as the corpus database is not distributed or sold,
its incorporation of material protected by copyright is defensible''.
Likewise, in my system I will be generating feature vectors for similarity matching, transforming the function of the ``derivative work'', and it is these that are stored in the database: ``[if] the defendant's work, although containing
substantially similar material, performs a different function than that of the
plaintiff's, the defense of fair use can be invoked'' \citep{Copyright1988}
\subsection{Competence}
On a more general level, it is important to undertake the project with good
professional conduct. I will aim to adhere to the British Computer Society's
Code of Good Practice and Code of Conduct, especially the following section:
\begin{quotation}
	\renewcommand{\labelenumi}{\arabic{enumi}.}
	\renewcommand{\labelenumii}{\roman{enumii}.}
	\begin{enumerate}
		\setcounter{enumi}{14}
		\item You shall not claim any level of competence that you do not possess. You shall only offer to do work or provide a service that is within your professional competence. You can self-assess your professional competence for undertaking a particular job or role by asking, for example,
		\begin{enumerate}
			\item am I familiar with the technology involved, or have I worked with similar technology before?
			\item have I successfully completed similar assignments or roles in the past?
			\item can I demonstrate adequate knowledge of the specific business application and requirements successfully to undertake the work?
		\end{enumerate}
	\end{enumerate}
\end{quotation}
% where my study during my degree lies
My project involves the study of Computational Music, one of the course courses
from my degree, and draws on knowledge gained from other courses such as
Data Mining, Artificial Intelligence and Multimedia Design and Applications.
As I have chosen a topic in the area of Music Informatics, I have the level of
competence required to undertake the project, and am able to research anything
which falls outside my area of knowledge.
\subsection{Falsifying of evidence}
% quote bcs
% my results are fully repeatable
\begin{quotation}
You shall not misrepresent or withhold information on the performance of products, systems or services, or take advantage of the lack of relevant knowledge or inexperience of others.
\end{quotation}



