\chapter{Method, Design, and Implementation (2500)}
\section{Theory \& Principles}
Explain the theory behind all this (psychoacoustics, DSP, etc.)
\section{System Overview}
% Talk about how this works (decode audio, extract features, measure similarity, interpolate) as a general overview
\section{Feature Extraction}
\begin{itemize}
	\item go into depth about feature extraction, talk about each extractor, how it works and what it models/looks for
\end{itemize}
\subsection{Standardized Moments}
For each feature extracted, I calculate the mean, variance, skewness and kurtosis over each three second block. It is these four values that are then used in the similarity measure, as it is unneccesary to store the output of each feature extractor for each window (and the amount of data would exceed what I set out in the specification, page \pageref{text:spec:requirement:data}). These values show how each feature for each window changes over the course of a block: variance describes how much the values change; skewness, how asymmetrical the data set is; and kurtosis, the `peakedness' or `flatness' of the data set \citep{Siegrist2007}.
\subsection{Signal Descriptors}
Signal descriptors are values which are calculated from the values of the signal itself. They tend to be values based on the periodicity or short-time amplitude of the signal. They are simple aggregate functions, and as of such are much less computationally expensive than the spectral descriptors.
\subsubsection{Zero Crossing Rate}
\begin{displaymath}
\mathit{Zero Crossing Rate} = \frac{1}{N_S} \sum_{i=0}^{N_S-2} sgn(s_i) \oplus sgn(s_{i+1})
\label{eq:zero_crossing_rate}
\end{displaymath}

The zero crossing rate is the number of times the signal passes from positive to negative or vice versa; it can be used as primitive pitch detection algorithm, and has been used successfully in the classification of percussive sounds \citep{Gouyon2000} and speech recognition[REF], due to the fact that it tends to be higher and unstable during the attack period of a sound \citep{Schwarz2004}.
\subsubsection{First Order Autocorrelation}
\begin{equation}
\mathit{FirstOrderAutocorrelation} = \frac{1}{N} \sum_{i=0}^{N-2} s_i \cdot s_{i+1}
\label{eq:first_order_autocorrelation}
\end{equation}

Autocorrelation is a measure of how well the signal matches a time-shifted version of itself, and hence shows the `noisyness' of the signal. \citep{Wikipedia2007}. As with zero crossing rate, this is useful to find attacks as well as being a measure of the harshness of the timbre of the sound.
\subsection{Spectral Descriptors}
Spectral descriptors use Fourier series (the output of a discrete Fourier transform); rather than working in the dimensions of amplitude and time, they output values from the dimensions of frequency and amplitude. Figure [REF] shows the data analysed for one block of audio for both the spectral and signal descriptors. Spectral descriptors, working in the frequency domain, aim to represent features of the timbre of the sound, with the higher statistical moments looking at the change in timbre over blocks.
\subsubsection{Linear Regression}
\begin{displaymath}
\beta = \frac{N_B \sum_{i=1}^{N_B} (F_i a_i) - \sum_{i=1}^{N_B} F_i \sum_{i=1}^{N_B} a_i}{N_B \sum_{i=1}^{N_B} {F_i}^2 - \sum_{i=1}^{N_B} F_i^2}
\label{eq:linear_regression}
\end{displaymath}

Linear regression finds the `line of best fit' through a set of data. I have used this statistical ??? to find the gradient ($\beta$) of the spectrum, which is a measure of the ratio of high-frequency to low-frequency component of the sound, and hence the timbre. With this feature the variance will most likely be a valuable similarity measure, as it will show the smoothness of timbre - long sustained notes will have low variation in linear regression, fast or noisy music will have a large variance.
\subsubsection{Centroid}
\begin{displaymath}
\mathit{SpectralCentroid} = \frac{\sum_{i=1}^{N_B} a_i F_i}{\sum_{i=1}^{N_B} a_i} 
\label{eq:spectral_centroid}
\end{displaymath}

The spectral centroid is like the centre of gravity of a set of data if it were imagined to be a 2D shape \citep{Park2004}; it represents the average frequency weighted by amplitude, and hence where most of the energy of the signal lies. Psychoacoustically it is a measure of the perceived `brightness' of the sound, providing a better estimation of a `bright' sound than pitch \citep{Schubert2004}, and tending to rise for more intense sounds.
\subsubsection{Intensity}
\begin{equation}
\mathit{SpectralIntensity} = \frac{1}{N} \sum_{i=0}^{N-1} (s_i)^2
\label{eq:spectral_intensity}
\end{equation}

Intensity gives a measure of the total amplitude in each window, giving an approximation of the perceived `loudness' of the sound. 
\subsubsection{Smoothness}
\begin{displaymath}
\mathit{SpectralSmoothness} = \sum_{i=1}^{N_B} 20 \log a_i - \frac{20\log a_{i-1} + 20\log a_{i} + 20\log a_{i+1}}{3}
\label{eq:spectral_smoothness}
\end{displaymath}

\subsubsection{Spread}
\begin{displaymath}
\mathit{SpectralSpread} = \sqrt{\frac{\sum_{i=1}^{N_B} a_i (F_i - \mathit{SpectralCentroid})^2}{\sum_{i=1}^{N_B} a_i}}
\label{eq:spectral_spread}
\end{displaymath}

\subsubsection{Dissymmetry}
\begin{equation}
\mathit{SpectralDissymmetry} = \sqrt{\frac{\sum_{i=0}^{N-1} a_i (f_i - \mathit{SpectralCentroid})^3}{\sum_{i=0}^{N-1} a_i}}
\label{formula_spectral_dissymmetry}
\end{equation}

\subsection{Commentary}

\begin{itemize}
	\item talk about reasons not to implement a tonne
	\item other possible extractors - MFCC, others used in Marsyas
\end{itemize}
\section{Similarity Measure}
\begin{itemize}
	\item problems of similarity measure
	\begin{itemize}
		\item having to normalise the vectors
		\item choices of algorithm for how to compare tracks - how does order of comparison affect results in similarity measure?
		\begin{itemize}
			\item 3sec blocks give bias to songs with beat aligned to block boundary (combatted by overall song features) - compare blocks in neighbouring groups to beat this
			\item structure (verse/chorus, intro, delay at start) will strongly affect a simple ordered comparison
			\item sorted \& total comparisons give `complete' comparison of track, though take O(n!) time to complete
			\item ``as of such I have implemented four techniques and will see which is best'' (har har)
		\end{itemize}
		\item NaN problems
	\end{itemize}
	\item feature weights - why have them, why train them
\end{itemize}
\section{Playlist Generation}
\begin{itemize}
	\item describe UI
	\item how it works
	\item ``a number of simple tools used in conjunction''
	\item UNIX quote?
\end{itemize}
\section{Weight Optimisation}
why have weights for the features, why train them
\begin{itemize}
	\item explain how the weight training works - listener testing app which plays three clips, the user selects which two are most similar
	\item why this helps us
	\item what we can use the results for
	\item other ways of collecting similar data
	\begin{itemize}
		\item tracks off the same album most likely similar - especially in the case of a compilation album
		\item last.fm and other services ``similar songs'' based on statistics can be used as a guide to the training on which songs are simliar
	\end{itemize}
\end{itemize}
\begin{itemize}
	\item system can discover important features for itself
	\item as it is not completely understood how we percieve sound, feature extractors only work as a basic model, therefore it is best to use lots of these models and let empirical evidence from testing `teach' the system the best models to use
	\item weights will help the system return results similar to the results given by the user - the perfect system would take the data learned from the user in the testing app and return identical results - we can test how similar the results of our system is to the user's output for a good measure of the efficacy of the system
\end{itemize}
\section{Implementation}
issues with implementation
\begin{itemize}
	\item Benefits of two-pass, analyse/search method: faster to generate playlists, library rebuilt far less often than playlists generated
	\item simple arithmetic to generate playlist = fast enough (maybe?) to run on portable device
	\begin{itemize}
		\item it is possible to precalcuate all distances between songs - size/speed tradeoff - take up more space on portable device but trivial to generate playlists on-the-go
	\end{itemize}
\end{itemize}
\section{Critique (of Design)}
\begin{itemize}
	\item features of the selected paradigm and justification
	\item advantages and disadvantages of my implementation / approach
	\item draw up a table in notes, easier to relate things
\end{itemize}
