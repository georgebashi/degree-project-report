\pagebreak
\marginsize{1cm}{1cm}{1cm}{2cm}
\chapter{Source Code}
\input{appendicies/code/highlight.sty}
\newcommand{\inputcode}[1]{
\section{#1}
\label{code:#1}
\small
\input{appendicies/code/#1.tex}
}
The source code of the system follows. Files are listed in compilation order, that is, each file depends only on the ones appearing earlier. The two \software{Ruby} scripts are themselves executable with a \software{Ruby} interpreter; the \software{C++} source must first be compiled and linked with \software{libpopt}, \software{FFTW3} and \software{libsndfile}, and has a run-time dependency on \software{GStreamer}.

The following versions of software were used in development:
\begin{description}
	\item[\software{GCC}] 4.1.2
	\item[\software{Ruby}] 1.8.5
	\item[\software{Ubuntu Linux}] 7.04
	\item[\software{libpopt}] 1.10
	\item[\software{FFTW}] 3.1.2
	\item[\software{libsndfile}] 1.0.16
	\item[\software{GStreamer}] 0.10.12
\end{description}
\inputcode{listenertest.rb}
\inputcode{evaluate.rb}
\inputcode{common.hh}
\inputcode{FFT.hh}
\inputcode{FFT.cc}
\inputcode{Features.hh}
\inputcode{Features.cc}
\inputcode{Song.hh}
\inputcode{Song.cc}
\inputcode{viewer.cc}
\inputcode{SoundFile.hh}
\inputcode{SoundFile.cc}
\inputcode{extractor.cc}
\inputcode{SongSet.hh}
\inputcode{SongSet.cc}
\inputcode{learner.hh}
\inputcode{learner.cc}
\inputcode{Playlist.hh}
\inputcode{Playlist.cc}
\inputcode{generator.hh}
\inputcode{generator.cc}

