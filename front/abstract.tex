\section*{Abstract}
%\note{motivation}
Thanks to the rise of the Internet and digital storage of audio, a new platform for delivery of audio has emerged. Libraries of hundreds of tracks can be acquired with relative ease and playback is at the click of a button. 
%\note{problem statement}
Despite this, methods of playback have remained fairly static, with the only real advance being \emph{playlists}, where a user can queue a selection of songs to later play back, a difficult and time-consuming process.
%\note{approach}
This project details a system for automatic interpolative playlist generation based upon a sub-symbolic music similarity measure, allowing users to select a set of `key' tracks, where songs chosen by the system are inserted between them to create a smooth, coherent transition.
%\note{results}
A statistical function for training and testing the system based on a user's answers to a `music quiz' is developed, the results of which show a similarity metric of reasonable accuracy, with at least 5 of 10 tracks reviewed rated as `very similar'. The subsequent playlist generator performs reasonably well, the output of which proved difficult for the test users to distinguish from a manually created playlist, showing that the system performs on par with manual creation.
%\note{conclusion}
The system implemented successfully meets the goals of reducing effort required to create `good' playlists and provides an interesting new way of listening to the contents of a media library.

% In evaluating PATS, Pauws describes coverage of the media library as important for good playlist generation. While no provision has been made for this in the system described, as the user is able to select key tracks to interpolate between they are able themselves to explore the library as they see fit. If a user selects a particular style or genre of music, it would negatively affect the coherence of the playlist to deliberately select tracks of a differing style, purely so they can explore `fresh' music. It is assumed that as the user has purchased all of the music in their library that they will have heard it previously to using a playlist generator; if not, they are free to choose unheard tracks to interpolate between, in order to explore the playlist. In evaluating PATS, Pauws describes coverage of the media library as important for good playlist generation. While no provision has been made for this in the system described, as the user is able to select key tracks to interpolate between they are able themselves to explore the library as they see fit. If a user selects a particular style or genre of music, it would negatively affect
\vspace{8cm}
