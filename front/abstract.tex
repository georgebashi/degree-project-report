The rise of the Internet and digital storage of audio has brought about a new platform for delivery of audio. Hundreds of tracks can be easily be acquired and played back at the click of a button. Nevertheless, methods of playback have remained fairly static, the only real advance being \emph{playlists} where a user can queue a selection of songs for play back, a difficult and time-consuming process. This project details a system for automatic playlist generation, where songs are automatically arranged between key tracks, creating a smooth, coherent transition of genre. The system is trained and tested based on answers to a `music quiz'. The output was comparable to a manually created playlist, and was therefore successful in reducing the effort required to create novel playlists and provides an interesting new way of listening to the contents of a media library.
