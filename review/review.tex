\subsection{Concepts}
Two key concepts exist in music similarity metric research; symbolic and sub-symbolic. Symbolic metrics work on the pre-existing metadata of the track, such as the artist and album names. Generally, statistics are recorded on the user's listening habits such as skipping behaviour \citep{Pampalk2005a}, play counts or user specified ratings, and these are used to determine songs that the user likes and would hence work well together in a playlist. Systems such as these are inherently flawed: if music is labelled incorrectly, has missing metadata, or no recorded statistics, it cannot be used with symbolic similarity metrics. Sub-symbolic systems aim to overcome this by operating directly on characteristics of the audio to either replace or complement the metadata.

\section{Review of Existing Systems}
Most current software for playlist generation uses symbolic techniques. Ruffle is a plugin to the Muine player, providing playlist generation or extension based on play counts and tagged genre. Apple's iTunes ``Party Shuffle'', which plays the entire media library in random order, biased towards highly-rated or often played songs. While popular, iTunes and Ruffle offer only a weighted-random approach to playlist generation, with no consideration of continuity of genre or mood.

Three key systems are described in the literature: \citet{Aucouturier2002a}, \citet{Logan2001} and \citet{Pauws2002}. Both Aucouturier and Logan describe sub-symbolic systems providing good results, however, Logan's performed poorly until symbolic data was added. \citet{Pauws2002} developed Personalised Automatic Track Selection (PATS), a symbolic system, based on clustering of artist, genre and year tags. The results of these systems show that playlist generators based on symbolic analysis have two shortcomings: having play counts and metadata as their only source of data, and the requirement of well tagged audio, and hence inability to generate playlists with untagged  audio. \citet{Aucouturier2002a} and \citet{Logan2001}, are a demonstrable improvement, despite having had mixed results. In summary, the quality of playlist generated depends heavily upon the accuracy of similarity metric.
