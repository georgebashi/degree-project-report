\chapter{Review of Existing Systems (1029/1000)}
\section{Outline}
Music Information Retrieval (MIR) is a relatively young topic involving the study of musical content analysis, modelling and archiving to provide semantic search and retrieval of music from large databases. MIR is already in use in a number of commercial projects, such as Shazam!, a service where songs can be identified over a telephone; MusicBrainz adds metadata to stored audio using fingerprinting techniques. 
In order to discover strengths and weaknesses in the area of playlist generation which the system will build upon, four current playlist generation software packages have been trialled by the author and reviewed.
\section{Concepts}
Before discussing the software implementations, it is important to define the terminology used in the area, and the distinction between the main methods of similarity measure and playlist generation. Two areas exist in music similarity metric research; symbolic and sub-symbolic. Symbolic metrics work on the preexisting metadata of the track, such as the artist and album names stored alongside, or using \software{ID3} tags inside, the audio data itself. Generally, statistics are recorded on the user's listening habits such as skipping behaviour \citep{Pampalk2005a}, play counts or user specified ratings, and these are used to determine songs that the user likes and would hence work well in a playlist together. Systems such as these are inherently flawed: if music is labelled incorrectly, has missing metadata, or no recorded statistics, it cannot be used with symbolic similarity metrics. Sub-symbolic systems aim to overcome this by operating directly on the audio data to measure similarity. The system `listens' to the music itself, and uses records characteristics of the audio to either replace or complement the data used in symbolic systems.

When operating on audio data, similarity metrics fall into two groups, again with the distinction of symbolic or sub-symbolic. Symbolic similarity metrics in this sense work on transcribed music, comparing the similarity of melodies on a note-by-note or tonality basis, similar to how a spelling corrector measures similarity between misspelled words and possible corrections. While this can provide great precision, it says nothing about how the music is perceived by the listener, and two songs of differing genre, while they may be similar would most likely be regarded as very dissimilar to a symbolic metric. Sub-symbolic metrics, on the other hand, model a listener's perception of a song, assigning values to attributes used to describe music such as brightness, energy and tempo.
\section{Software}
Most (if not all) current publicly available software for playlist generation use symbolic techniques to infer similarity and generate playlists. Ruffle is a plugin to Muine, a popular Linux audio player, which provides the ability to either generate playlists or extend user-created ones based solely on play counts and the genre tagged in songs. It works well when a large amount of data has been collected and provides an easy-to-use interface to playlist generation. Apple have implemented a similar system in their iTunes software called ``Party Shuffle''. The entire media library is played in random order, biased towards highly-rated or often played songs, and allows the user to remove tracks from the upcoming playlist at will, replacing these with other suggestions. While popular, iTunes and Ruffle offer only a weighted-random approach to playlist generation, with no consideration of continuity of genre or mood.

The Filter is a newly released plugin for Apple's iTunes which amasses play count data from all of its users into a central database. Using such metadata as tempo and genre, along with user's listening habits, it improves on iTunes' Party Shuffle by introducing the concept of continuity. The user is asked to specify a few `seed' tracks, which are then looked up in the on-line database and used to set the mood/genre for the generated playlist.
\section{Literature}
\citet*{Aucouturier2002a} suggest a similarity measure based on a Gaussian mixture model (GMM) of Mel frequency cepstral coefficients (MFCCs). Implemented for playlist generation by \citet{Schnitzer2003}, it works well to measure similarity between tracks returning a number of good results. This technique was later implemented for the CUIDADO project by \citet*{Aucouturier2003}, using an adaptive constraint satisfaction algorithm based on four constraints of no repetition, a desired length, continuity in genre and progression in the theme of the playlist. The benefits of this technique are its high scalability and accuracy, working well with a library of around 200,000 tracks.

\citet{Logan2001} present an alternative similarity metric based on k-means clustering of spectral features, using ``song trajectories'' to create interesting playlists. To test the playlist generation using this technique, a `baseline' playlist was created of the most similar $n$ songs, which was then compared to one using song trajectories. The outcome of this testing showed that it actually performed worse than the baseline playlist, but more promising results were shown when symbolic data was added to guide the metric.

A fully symbolic playlist generation system is described by \citet{Pauws2002}. Personalised Automatic Track Selection (PATS) is a system developed to ease the selection of music for a given context of use, such as listening to soft or lively music. It uses metadata attribute similarity along with a dynamic clustering technique to generate playlists, which were described to have higher coverage of the library, higher precision in selecting similar tracks, and a higher user rating than a baseline playlist of random tracks. This system has a unique method of training, in that users can specify tracks to be dropped from the playlist and using a decision tree, the system trains the weights of the similarity metric. 
\section{Review and Identification of Problem Areas}
Playlist generators based on symbolic analysis such as \software{Ruffle} and \software{The Filter} have two shortcomings: the fact that play counts for each track are their only source of data for statistical analysis, and the reliance on well tagged audio for the similarity metric. \software{The Filter} improves upon this by correlating data between all its users, but still is unable to perform any playlist generation where the songs are completely untagged or `unseen' in the database. Additionally, these techniques rely on the user only listening to artists similar to each other, not accounting for users having wide tastes in music or even simply randomly listening to tracks from the whole library, as the order of tracks played is recorded.

Two systems (Aucouturier and Logan's) have been reviewed detailing systems which are based on sub-symbolic analysis of the audio data in order to overcome the problems with existing symbolic systems. While these have had mixed results, it is a promising area of study and would, depending on the accuracy of the similarity metric developed, provide a more suitable playlist generation without the shortcomings of depending on accurate symbolic data.

In order to comprehensively test the system designed, it will be tested in a method similar to PATS; as well as statistical testing showing the performance of the similarity metric, a small group of users will be interviewed to gain insight into how well the system performs overall and the quality of the playlists generated from the standpoint of the target audience.
