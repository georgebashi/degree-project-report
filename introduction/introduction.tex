%%%%%%%%%%%%%%%%%%%%%%%%%%%%%%%%%%%%%%%%%%%%%%%%%%%%%%%%%%%%%%%%%%
% NEED TO TALK ABOUT THE IMPORTANCE OF SIMILARITY MEASURE TO     %
% THE PERFORMANCE OF PLAYLIST GENERATION, AND HOW INTERPOLATION  %
% ALLOWS THE SIMLIARITY MEASURE TO BE LESS RELIABLE              %
%%%%%%%%%%%%%%%%%%%%%%%%%%%%%%%%%%%%%%%%%%%%%%%%%%%%%%%%%%%%%%%%%%
\chapter{Introduction (500)}
\begin{itemize}
	\item rationale
	\item purpose
	\item issues
	\item context
	\item relevance
	\item structure of dissertation
\end{itemize}
\section{Introduce Topic of Playlist Generation \& Uses}
\begin{itemize}
	\item Select MP3s
	\item Makes Playlist by analysing tracks in the user's media library
	\item feature extraction for music similarity measure
	\item Eases creating `good' playlists
\end{itemize}
\section{Rationale \& Context of Use}
\begin{itemize}
	\item basis of my work --- why i'm doing it
	\item Parties
	\item Personalised Radio
	\item `Intelligent' random
\end{itemize}
\section{Issues}
Creating a `good' playlist of songs is a time consuming process, requiring the user to select and preview each individual track, and manually adding them to a flat list of songs to play.
\section{Purpose \& Relevance}
Using music information retrieval techniques, I hope to build a system which will ease the creation of novel and interesting playlists.

Talk about problems of MIR (transcription, cultural, semantic, etc.)
\section{Structure of Dissertation}
I will begin by reviewing similar current research and software in the area of playlist generation and music similarity measure, and go on to suggest and implement a method for interpolative playlist generation. After implementing some methods of similarity measure, I will test these on a corpus of tracks from example music libraries, and evaluate the performance of the implemented methods.
