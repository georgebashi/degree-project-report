\chapter{Testing (626/500)}
Testing the functioning, performance and accuracy of the system allows for evaluation as to how well the system meets its specification and purpose. Three methods have been employed for testing the areas of the system, and will help identify strengths, weaknesses, and areas for improvement later in the analysis.
\section{Functional Testing}
\label{text:testing:functional}
\note{Method Used}
To show correct operation of the system and give examples of generated playlists, excerpts of the output of the program are given along with some example queries and their output.
\note{Results}
\subsection{\executable{extractor}}
For this test, the feature extractor was run on three tracks from \emph{Kasabian}'s album, \emph{Empire}.\\
Command:
\begin{verbatim}
./extractor ../test/Kasabian/Empire/Empire-04-Me\ Plus\ One.mp3 \
    ../test/Kasabian/Empire/Empire-06-Apnoea.mp3 \
    ../test/Kasabian/Empire/Empire-08-Stuntman.mp3
\end{verbatim}
Output:
\begin{verbatim}
../test/Kasabian/Empire/Empire-04-Me Plus One.mp3: decode, analyse, done (83 blocks)
../test/Kasabian/Empire/Empire-06-Apnoea.mp3: decode, analyse, done (60 blocks)
../test/Kasabian/Empire/Empire-08-Stuntman.mp3: decode, analyse, done (181 blocks)
\end{verbatim}

The three tracks were processed, and the features written out to files on disk, 12KB, 8KB and 24KB in size respectively.
\subsection{\executable{viewer}}
To show the operation of \executable{viewer}, it was run on one of the feature vectors outputted by \executable{extractor} above, to show the feature blocks (output summarised).\\
Command:\\
\verb"./viewer -b ../test/Kasabian/Empire/Empire-04-Me\ Plus\ One.mp3.vec"\\
Output:
\begin{alltt}
"Block","Mean of Zero Crossing Rate","Variance of Zero Crossing Rate","Skewness of Zero\ldots
1,0.0482357,0.00052548,0.251996,2.37784,0.0329947,0.000534037,1.97615,7.77386,0.00112283,\ldots
2,0.047806,0.000511404,0.0219624,1.97833,0.0325186,0.000534909,1.70473,6.22687,0.00110232,\ldots
3,0.0476497,0.000567088,0.465436,2.74142,0.0310403,0.000429462,1.70945,6.72855,0.00104859,\ldots
4,0.0786003,0.00558258,2.66379,10.0214,0.0388259,0.000730849,1.59315,6.35009,0.00128985,\ldots
5,0.0915951,0.00356234,1.46339,4.70598,0.0555432,0.00101183,1.40148,5.28068,0.00183747,\ldots
\end{alltt}


\subsection{\executable{learner}}
After all songs were processed and the listener test was performed to gain around 100 results, the learner was run to calculate a set of weights:\\
Command:
\begin{verbatim}
./learner --dir=../test
\end{verbatim}
Output:
\begin{verbatim}
1413 songs loaded
{ 0.613092, 0.205134, 0.533735, 0.110462}, 
{ 0.306893, 0.132636, 0.453016, 0.0505156}, 
{ 0.321113, 0.183988, 0.485224, 0.0169958}, 
{ 0.459398, 0.206291, 0.48236, 0.260304}, 
{ 0.279731, 0.121066, 0.449003, 0.0465693}, 
{ 0.677096, 1, 0.786617, 0.545458}, 
{ 0.997773, 0.680246, 0.306549, 0.0242729}, 
{ 0.718418, 0.303971, 0.29711, 0}
\end{verbatim}

These weights were then pasted into the code of \executable{generator} and recompiled to be used in subsequent testing.
\pagebreak
\subsection{\executable{generator}}
\subsubsection{Similarity Measure}
For the first test, the playlist generator was run to provide the 50 most similar tracks to the given song, \emph{Quick and to the Pointless} by \emph{Queens of the Stone Age}. NB: there was only one \emph{Queens of the Stone Age} track in the media library, hence no others appearing in the results. The distance for each track appears in brackets after the file name.\\
Command:
\begin{verbatim}
./generator --verbose --similar=50 --comparison=sorted --dir=../test \
../test/Queens\ of\ the\ Stone\ Age/Rated\ R/Rated\ R-07-Quick\ and\ to\ the\ Pointless.mp3.vec
\end{verbatim}
Output: \small \emph{(Line numbers have been added for ease of reference)}
\begin{listing}{1}
../test/nofx/The Greatest Songs Ever Written (By Us)/The Greatest Songs Ever Written (By Us)-24-It's My Job to Keep Punkrock Elite.ogg (0.297521)
../test/Rancid/Rancid/Rancid-17-Meteor of War.mp3 (0.305863)
../test/The Distillers/Sing Sing Death House/Sing Sing Death House-05-Sing Sing Death House.mp3 (0.310962)
../test/Various Artists/BYO Split Series, Volume V/BYO Split Series, Volume V-03-Alkaline Trio-Dead and Broken.mp3 (0.314302)
../test/Rancid/Rancid/Rancid-06-Poison.mp3 (0.318846)
../test/Alkaline Trio/Remains/Remains-13-Dead and Broken.mp3 (0.320534)
../test/The Distillers/Sing Sing Death House/Sing Sing Death House-10-Desperate.mp3 (0.322074)
../test/nofx/The Greatest Songs Ever Written (By Us)/The Greatest Songs Ever Written (By Us)-25-Franco Un-American.ogg (0.325217)
../test/Junior Senior/Hey Hey My My Yo Yo/Hey Hey My My Yo Yo-08-Ur a Girl.mp3 (0.326607)
../test/Various Artists/Mob Mentality/Mob Mentality-02-The Business-In the Streets of London.mp3 (0.327603)
../test/Rancid/Rancid/Rancid-07-Loki.mp3 (0.331062)
../test/Rancid/Rancid/Rancid-13-Not to Regret.mp3 (0.33118)
../test/Various Artists/Punk-O-Rama, Volume 10/Punk-O-Rama, Volume 10-27-Roger Miret and the Disasters-Riot, Riot, Riot.ogg (0.332419)
../test/Various Artists/Mob Mentality/Mob Mentality-08-The Business-Borstal Boys.mp3 (0.336915)
../test/Alkaline Trio/Goddamnit/Goddamnit-02-Cop.mp3 (0.341494)
../test/Rancid/Rancid/Rancid-09-Rejected.mp3 (0.34587)
../test/Various Artists/Punk-O-Rama, Volume 10/Punk-O-Rama, Volume 10-20-Rancid-White Knuckle Ride.ogg (0.346863)
../test/Rancid/Rancid/Rancid-14-Radio Havana.mp3 (0.347537)
../test/nofx/The Greatest Songs Ever Written (By Us)/The Greatest Songs Ever Written (By Us)-06-Bleeding Heart Disease.ogg (0.352109)
../test/The Distillers/Coral Fang/Coral Fang-05-Coral Fang.mp3 (0.354196)
../test/Alkaline Trio/Crimson (bonus disc)/Crimson (bonus disc)-01-Time to Waste (demo).mp3 (0.355211)
../test/Rancid/and Out Come the Wolves/and Out Come the Wolves-12-She's Automatic.mp3 (0.355757)
../test/Alkaline Trio/Good Mourning/Good Mourning-03-One Hundred Stories.mp3 (0.355769)
../test/Alkaline Trio/Crimson (bonus disc)/Crimson (bonus disc)-05-Dethbed (demo).mp3 (0.356513)
../test/Alkaline Trio/Maybe I'll Catch Fire/Maybe I'll Catch Fire-08-She Took Him to the Lake.mp3 (0.358377)
../test/Various Artists/BYO Split Series, Volume V/BYO Split Series, Volume V-08-One Man Army-The Hemophiliac.mp3 (0.360446)
../test/Alkaline Trio/Good Mourning/Good Mourning-07-Fatally Yours.mp3 (0.359591)
../test/Alkaline Trio/Good Mourning/Good Mourning-14-Old School Reasons.mp3 (0.360203)
../test/Four Tet/DJ-Kicks Four Tet/DJ-Kicks Four Tet-09-Animal Collective-Baby Day.mp3 (0.357063)
../test/Various Artists/BYO Split Series, Volume III/BYO Split Series, Volume III-07-NOFX-I'm the One.mp3 (0.360754)
../test/Alkaline Trio/Good Mourning/Good Mourning-06-Emma.mp3 (0.361064)
../test/System of a Down/Mezmerize/Mezmerize-03-Revenga.mp3 (0.361441)
../test/nofx/The Greatest Songs Ever Written (By Us)/The Greatest Songs Ever Written (By Us)-05-Murder the Government.ogg (0.36334)
../test/Alkaline Trio/Remains/Remains-16-Wait for the Blackout.mp3 (0.364695)
../test/Alkaline Trio/Crimson (bonus disc)/Crimson (bonus disc)-06-Settle for Satin (demo).mp3 (0.364749)
../test/Rancid/Rancid/Rancid-08-Blackhawk Down.mp3 (0.365072)
../test/Basement Jaxx/Rooty/Rooty-01-Romeo.flac (0.365695)
../test/Alkaline Trio/Alkaline Trio/Alkaline Trio-09-For Your Lungs Only.mp3 (0.365721)
../test/Rancid/Rancid/Rancid-05-Another Night.mp3 (0.366004)
../test/Rancid/Rancid/Rancid-18-Dead Bodies.mp3 (0.366937)
../test/System of a Down/Mezmerize/Mezmerize-02-B.Y.O.B..mp3 (0.367975)
../test/Various Artists/Punk-O-Rama, Volume 10/Punk-O-Rama, Volume 10-14-Converge-Black Cloud.ogg (0.368253)
../test/Goldfinger/The Best of Goldfinger/The Best of Goldfinger-03-Miles Away.mp3 (0.368692)
../test/Rancid/Rancid/Rancid-21-Reconciliation.mp3 (0.370021)
../test/Saves the Day/Can't Slow Down/Can't Slow Down-10-Hot Time in Delaware.mp3 (0.372484)
../test/The Distillers/Sing Sing Death House/Sing Sing Death House-04-The Young Crazed Peeling.mp3 (0.373146)
../test/Rancid/Rancid/Rancid-11-Antennas.mp3 (0.373572)
../test/Rancid/Rancid/Rancid-10-Corruption.mp3 (0.373799)
../test/Rancid/Rancid/Rancid-10-Injury.mp3 (0.374139)
../test/Alkaline Trio/Remains/Remains-09-Old School Reasons.mp3 (0.37425)
\end{listing}


\subsubsection{Playlist Generation}
To test the playlist generation, two playlists were generated to highlight difficulties in creating `good' playlists with the technique described: a short playlist between two tracks disparate in genre (as it is difficult to create a smooth progression with only a small amount of tracks), and a long playlist between two similar songs (as finding tracks by \emph{different artists} which can create a smooth transition is difficult).

First, the short playlist, between a fast-paced aggressive track by \emph{System of a Down}, and a slow, calm and atmospheric track by \emph{Four Tet}.\\
Command:
\begin{verbatim}
./generator --comparison=sorted --interpolate=3 --dir=../test
    ../test/System\ of\ a\ Down/Mezmerize/Mezmerize-04-Cigaro.mp3.vec
    ../test/Four\ Tet/Pause/Pause-08-No\ More\ Mosquitoes.mp3.vec
\end{verbatim}
Output: \small \emph{(Line numbers have been added for ease of reference)}
\begin{listing}{1}
../test/System of a Down/Mezmerize/Mezmerize-04-Cigaro.mp3
../test/Saves the Day/Sound the Alarm/Sound the Alarm-01-Head for the Hills.mp3
../test/RJD2/Since We Last Spoke/Since We Last Spoke-02-Exotic Talk.mp3
../test/Basement Jaxx/Rooty/Rooty-02-Breakaway.flac
../test/Four Tet/Pause/Pause-08-No More Mosquitoes.mp3
\end{listing}


Second, the long playlist, between two tracks by \emph{Saves the Day}, both melodic, up-beat songs with vocals.\\
Command:
\begin{verbatim}
./generator --comparison=sorted --interpolate=15 --dir=../test ../test/Saves\ the\ Day/In\ Reverie/In\ Reverie-05-In\ Reverie.mp3.vec ../test/Saves\ the\ Day/In\ Reverie/In\ Reverie-01-Anywhere\ With\ You.mp3.vec
\end{verbatim}
Output: \small \emph{(Line numbers have been added for ease of reference)}
\begin{listing}{1}
../test/Saves the Day/In Reverie/In Reverie-05-In Reverie.mp3
../test/Every Time I Die/Hot Damn!/Hot Damn!-06-Floater.mp3
../test/The Prodigy/The Dirtchamber Sessions, Volume One/The Dirtchamber Sessions, Volume One-02-Various Artists-Section 2: Bug Powder Dust  Pump Me Up  How High  Poison  Been Caught Stealing  I Get Wrecked.flac
../test/Alkaline Trio/Alkaline Trio/Alkaline Trio-08-Cooking Wine.mp3
../test/Gescom/Minidisc/Minidisc-82-Hemiplegia 2.mp3
../test/Collide/Some Kind of Strange/07 Mutation.flac
../test/Mindless Self Indulgence/Frankenstein Girls Will Seem Strangely Sexy/Frankenstein Girls Will Seem Strangely Sexy-12-Holy Shit.mp3
../test/Four Tet/Dialogue/Dialogue-05-Misnomer.mp3
../test/Silverstein/Discovering the Waterfront/Discovering the Waterfront-09-Already Dead.mp3
../test/Transplants/Haunted Cities/Haunted Cities-11-I Want It All.mp3
../test/Goldfinger/The Best of Goldfinger/The Best of Goldfinger-04-Superman.mp3
../test/Squarepusher/Big Loada/Big Loada-10-Significant Others.mp3
../test/88 Fingers Louie/Behind Bars/Behind Bars-05-Smart Enough to Run.mp3
../test/The Sainte Catherines/Dancing for Decadence/Dancing for Decadence-06-Emo-Ti-Cons: Punk Rock Experts.mp3
../test/Rancid/Let's Go/Let's Go-13-Ghetto Box.mp3
../test/The Avalanches/Since I Left You/Since I Left You-02-Stay Another Season.mp3
../test/Saves the Day/In Reverie/In Reverie-01-Anywhere With You.mp3
\end{listing}


It can be seen that in both playlists, \executable{generator} found difficulty in generating a `good' playlist while still adhering to the no-repeated-artists rule.
\pagebreak
\section{Automated Testing}
As discussed previously (in Method, page \pageref{text:method:weight_optimisation}), it is possible to use the output from the listener test to rate the system's accuracy before and after training. For each of the results, \executable{evaluate.rb} generates a playlist with each of the four comparison methods. The average position of the similar and not-similar tracks are then taken, along with their standard deviation, and the difference between placement of the similar and not-similar track. Low values for the average placement of similar tracks and standard deviation, as well as high values for the difference between similar and not-similar would indicate an accurate system.
\subsection{Results}
Figure \ref{graph:metric-comparison} shows a comparison of the four track comparison methods detailed on page \ref{text:method:comparison_methods}, before any training (all weights set to 1). These results show that the sorted comparison performed the best overall (higher average placement of similar track, less deviation from the mean, and a high difference between similar and not-similar tracks.

After initial testing, it seemed that training the system actually provided \emph{worse} accuracy. To investigate this further, I modified the source code of \executable{learner} to only train on the albums in the user's media library or on the answers from the listener test, and re-ran the test (figure \ref{graph:training-method-comparison}). The general trend from the data shows that the more data used in training, the worse the accuracy of the similarity metric.
 
Table \ref{table:testing_data} shows a summary of data from all tests.
\begin{table}[hp]
	\caption{Testing Data}
	\label{table:testing_data}
	\centering
\footnotesize \begin{tabular}{|l|l|l|l|l|l|l|}
\cline{3-6}
\multicolumn{1}{c}{} &  & \multicolumn{2}{c|}{\textbf{Similar}} & \multicolumn{2}{c|}{\textbf{Not Similar}} & \multicolumn{1}{l}{} \\ 
\hline
\multicolumn{1}{|c|}{\textbf{Training}} & \multicolumn{1}{c|}{\textbf{Metric}} & \multicolumn{1}{c|}{\textbf{Mean}} & \multicolumn{1}{c|}{\textbf{StDev}} & \multicolumn{1}{c|}{\textbf{Mean}} & \multicolumn{1}{c|}{\textbf{StDev}} & \multicolumn{1}{c|}{\textbf{Difference}} \\ 
\hline
\multicolumn{1}{|c|}{\textbf{Untrained}} & Ordered & 469.23 & 354.65 & 807.08 & 420.82 & 337.85 \\ 
\cline{2-7}
\multicolumn{1}{|c|}{} & Ordered Area & 484.39 & 370.1 & 827.74 & 428.44 & 343.35 \\ 
\cline{2-7}
\multicolumn{1}{|c|}{} & Sorted & 463.81 & 338.19 & 805.23 & 423.69 & 341.42 \\ 
\cline{2-7}
\multicolumn{1}{|c|}{} & Exhaustive & 474.35 & 353.99 & 803.4 & 423.26 & 329.05 \\ 
\hline
\multicolumn{1}{|c|}{\textbf{Both}} & Ordered & 493.97 & 363.67 & 799.74 & 426.5 & 305.77 \\ 
\cline{2-7}
\multicolumn{1}{|c|}{\textbf{Albums}} & Ordered Area & 507.47 & 379.27 & 822.61 & 434.64 & 315.15 \\ 
\cline{2-7}
\multicolumn{1}{|c|}{\textbf{and}} & Sorted & 481.26 & 344.64 & 802.24 & 425.42 & 320.98 \\ 
\cline{2-7}
\multicolumn{1}{|c|}{\textbf{Answers}} & Exhaustive & 499.44 & 364.79 & 796.08 & 428.32 & 296.65 \\ 
\hline
\multicolumn{1}{|c|}{\textbf{Albums}} & Ordered & 493.97 & 363.67 & 799.76 & 426.51 & 305.79 \\ 
\cline{2-7}
\multicolumn{1}{|c|}{} & Ordered Area & 507.47 & 379.27 & 822.61 & 434.64 & 315.15 \\ 
\cline{2-7}
\multicolumn{1}{|c|}{} & Sorted & 481.19 & 344.7 & 802.24 & 425.42 & 321.05 \\ 
\cline{2-7}
\multicolumn{1}{|c|}{} & Exhaustive & 499.44 & 364.79 & 796.06 & 428.31 & 296.63 \\ 
\hline
\multicolumn{1}{|c|}{\textbf{Answers}} & Ordered & 488.85 & 358.98 & 798.69 & 426.93 & 309.84 \\ 
\cline{2-7}
\multicolumn{1}{|c|}{} & Ordered Area & 500.52 & 373.69 & 821.61 & 434.23 & 321.1 \\ 
\cline{2-7}
\multicolumn{1}{|c|}{} & Sorted & 473.39 & 344.12 & 809.07 & 424.57 & 335.67 \\ 
\cline{2-7}
\multicolumn{1}{|c|}{} & Exhaustive & 495.37 & 360.73 & 794.37 & 429.72 & 299 \\ 
\hline
\multicolumn{1}{|c|}{\textbf{New}} & Ordered & 440.6
 & 349.36
 & 811.08
 & 406.04
 & 370.48
 \\ 
\cline{2-7}
\multicolumn{1}{|c|}{\textbf{Algorithm}} & Ordered Area & 445.74
 & 445.74
 & 825.15
 & 825.15
 & 379.4
 \\ 
\cline{2-7}
\multicolumn{1}{|c|}{} & Sorted & 438.15
 & 438.15
 & 806.5
 & 410.97
 & 410.97
 \\ 
\cline{2-7}
\multicolumn{1}{|c|}{} & Exhaustive & 445.02
 & 445.02
 & 819.97
 & 398.63
 & 398.63
 \\ 
\hline
\end{tabular}
\end{table}

\newcommand{\graph}[2]{
\begin{figure}[hp]
	\caption{#2}
	\includegraphics[width=\textwidth]{testing/graphs/#1}
	\label{graph:#1}
\end{figure}}
\graph{metric-comparison}{Comparison of Similarity Metrics}
\graph{training-method-comparison}{Comparison of Training Methods}
%\graph{training-algorithm-comparison}{Comparison of Training Algorithms}
\pagebreak
\section{User Testing}
\begin{itemize}
	\item Method used
	\item Results
	\item Summary
\end{itemize}
\section{Summary}

