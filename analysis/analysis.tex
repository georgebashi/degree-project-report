\chapter{Analysis (991/1500)}
\begin{comment}
	\item Subject the findings to scrutiny as to what they might mean
	\item Discussion and analysis of the theories, ideas, issues and challenges noted earlier in the writing. Do you have both intended and unintended outcomes?
	\item A `making sense' of the findings by considering their implications for the questions raised
	\item A critique of the research method (data collection tools?) used and their validity and reliability
\end{comment}
\section{Accuracy}
Following user testing (page \pageref{text:testing:user}), it was determined that despite some deficiencies in track selection the playlists generated were generally of high quality; two of the three users found it difficult to distinguish the manually created and generated playlists, showing that the output of the system is on par with manual track selection. The main reason the generated playlists were identified was due to the lack of analysis of mood or tempo. To overcome this, a feature determining the tracking could be implemented, as well as a mood discriminating technique such as detailed by \citet*{Liu2003} or a genre classification algorithm such as \citet*{Basili2004}.

Satisfactory results were produced by the similarity metric described, with at least half of the top ten results for each track in testing being rated very similar. In testing, a small number of disparate songs were returned in results, usually appearing in playlists as one or two `bad' tracks per 17-track playlist, suggesting a greater amount of training required. As time for the project was limited a small training set of around 100 tracks was collected, and would need to be vastly increased for any future development, though training based on this set showed promising results in testing. The automated testing method used, while suitable for a heuristic evaluation of song comparison algorithms, could be further improved using a corpus of music pre-analysed by an expert musicologist \citep{Muellensiefen2004}, affording further insight into perceptual similarity comparisons.

Of the four track comparison algorithms compared, the sorted comparison performed by far the best. Although an unexpected result, the algorithm can quickly compare the most similar sections of a track, which was evidenced in user C's evaluation of the playlist generation (page \pageref{text:testing:user:blind:c}) and provide better similarity measure. Further testing may indicate that an algorithm which accounts for song structure, perhaps based on techniques described in \citet{Bruderer2006}, may work better in the playlist generation stage as all three test users remarked on structure when commenting on similarity.

The playlist generation technique described overall works well; while there are problems with inconsistent genre, mood and tempo (caused by the similarity metric), the generation of playlists with good continuity, variance and progression while avoiding the repetition of artists is achieved. The system aims to optimise the overall score of the playlist by keeping tracks which work best in the playlist, choosing to remove tracks where the artist is repeated and the score is low. Other techniques based on constraint satisfaction \citep{Vossen2005} and Gaussian mixture modelling \citep{Aucouturier2003} have been implemented to optimise the total score of the playlist which is not considered in the algorithm described; the algorithm designed however is much less intensive, and hence better suits the design parameters.
\section{Efficacy}
As a whole, the system meets all design requirements set out in the specification, page \pageref{text:spec}. Making careful and informed choices about the method used helped to plan the system such that it was easily extensible and allowed for detailed in-depth testing.

The first research requirement of providing verifiability and review was achieved by providing a binary file viewer and verbose output options. \executable{generator} prints scores used in the similarity metric that proved valuable in testing, allowing analysis of any anomalous tracks in the playlist.

The command-line interface used did its duty to provide an interface that allowed creation of simple test scripts. Written in \software{Ruby} to speed development time, \executable{listenertest.rb} was able to analyse the similarity metric in bulk and return useful statistical information for later analysis. Although clumsy for an end-user, the command-line interface is well documented (\texttt{--help} can be passed to any of the programs to obtain usage information) and as of such would be simple for a future developer to pick up and work on.

Moving on to the user requirements, the system processes audio files in a relatively short amount of time (5--8 seconds per track on the development machine), owing to the careful choice and avoidance of overly expensive features such as MFCCs, which also helped achieve the requirement of small data storage. Playlists are generated very quickly, requiring only a matter of seconds to generate a playlist of reasonable length from a library of 1400 tracks.
\section{Weaknesses}
A number of weaknesses in the algorithms were detailed in testing (page \pageref{text:testing}), such as the discovery and revision of the learning algorithm. In general terms, weaknesses were found in the similarity measure due to the lack of awareness of tempo, mood and genre. Adding specific features (and perhaps replacing the worst performing, page \pageref{text:testing:functional:learner}) would help mitigate this, though as by nature the metric is an estimation, and hence it would never be possible to create a flawless model of perceptual similarity.

In evaluating PATS, Pauws describes coverage of the media library as important for good playlist generation. While no provision has been made for this in the system described, as the user is able to select key tracks to interpolate between they are able themselves to explore the library as they see fit. If a user selects a particular style or genre of music, it would negatively affect the coherence of the playlist to deliberately select tracks of a differing style, purely so they can explore `fresh' music. It is assumed that as the user has purchased all of the music in their library that they will have heard it previously to using a playlist generator; if not, they are free to choose unheard tracks to interpolate between, in order to explore the playlist.

A weakness in testing can be found in the data set used. As the data set is an example user's media library, tracks will tend to be of relatively similar genre due to the user's tastes; a fairer test would involve collecting an equal number of examples of music from all main genres of music. The difficulty in obtaining such a corpus lies in the copyright issues discussed in the professional considerations (page \pageref{text:spec:profcons}); obtaining an even cross-section of music would mean purchasing a large amount of music purely for the purpose of evaluating the system, which available resources do not allow.
\begin{comment}
	\item NaN problems
	\item requirement of pre-processing tracks, speed of processing
	\item similarity measure could have higher accuracy
	\item similarity measure doesn't consider mood or genre
	\item explain why the system doesn't have provisions for exploring the library (variance)
	\item Album effect, \citet*{Kim2006}
	\item evaluating similarity measures \citet*{Aucouturier2004}
	\item Automatic Music Transcription, transcription problem, symbolic similarity measure from subsymbolic data \citet*{Aucouturier2004}
	\item Structure \citet*{Bruderer2006}
	\item Refinement, training and optimisation, \citet*{Muellensiefen2004}
	\item Local search, constraint satisfaction \citet*{Vossen2005}
\end{comment}
