\chapter{Analysis (1500)}
\begin{itemize}
	\item Subject the findings to scrutiny as to what they might mean
	\item Discussion and analysis of the theories, ideas, issues and challenges noted earlier in the writing. Do you have both intended and unintended outcomes?
	\item A `making sense' of the findings by considering their implications for the questions raised
	\item A critique of the research method (data collection tools?) used and their validity and reliability
\end{itemize}
\section{Accuracy}
\note{how well the system performs as a similarity measure and subsequent playlist generator}
\section{Efficacy}
\note{how well the system as a whole meets the specifications, either reiterate them or refer back}
\section{Weaknesses}
As the data set used in testing is an example user's media library, tracks will tend to be of relatively similar genre due to the user's tastes; a fairer test would involve collecting an equal number of examples of music from all main genres of music.
\begin{itemize}
	\item NaN problems
	\item requirement of pre-processing tracks, speed of processing
	\item similarity measure could have higher accuracy
	\item similarity measure doesn't consider mood or genre
	\item explain why the system doesn't have provisions for exploring the library (variance)
\end{itemize}
\begin{itemize}
	\item Album effect, \citet*{Kim2006}
	\item evaluating similarity measures \citet*{Aucouturier2004}
	\item Automatic Music Transcription, transcription problem, symbolic similarity measure from subsymbolic data \citet*{Aucouturier2004}
	\item Structure \citet*{Bruderer2006}
	\item Refinement, training and optimisation, \citet*{Bruderer2006}
	\item Local search, constraint satisfaction \citet*{Vossen2005}
\end{itemize}
